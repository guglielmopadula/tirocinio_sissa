\documentclass{beamer}
\title{Geometry Morphing}
\usetheme{Copenhagen}
\usepackage{graphicx}
\usepackage{tikz}
\usetikzlibrary{positioning,arrows.meta,quotes}
\usetikzlibrary{shapes,snakes}
\usetikzlibrary{bayesnet}
\tikzset{>=latex}
\tikzstyle{plate caption} = [caption, node distance=0, inner sep=0pt,
below left=5pt and 0pt of #1.south]
\begin{document}
\frame{\titlepage}
\begin{frame}{The objective}
The objective of this internship is to morph some tetrahedral meshes preserving some properties (for now, the volume). We want to achieve this in two ways:
\begin{itemize}
\item Using (semi-discrete) Optimal Transport
\item Using Variational Autoencoders
\end{itemize}
\end{frame}
\begin{frame}{Some notation}
Let $A$ a tetrahedral mesh, composed by $N$ tetrahedrals, then $A_{i}$ will denote the volume matrix of the $i$-th tetrahedra:
\\
$\begin{array}{|ccc|}
a_{1}-d_{1} & b_{1}-d_{1} & c_{1}-d_{1} \\
a_{2}-d_{2}& b_{2}-d_{2} & c_{2}-d_{2} \\
a_{3}-d_{3} & b_{3}-d_{3} & c_{3}-d_{3} \end{array}$
\\
where $a,b,c,d$ are the points of the $i$-th tetrahedral. It is called volume matrix because the volume of the tetrahedra is $\frac{|\det(A_{i})|}{6}$, by swapping $a,b,c,d$ we can assume that $det(A_{i})>0$.
The volume of the mesh so can be calculated with $\frac{\sum_{i=1}^{N}det(A_{i})}{6}$
\end{frame}

\begin{frame}{Optimal Transport map I}
Given $\Omega$ a Borel set and two measures $\mu$ and $\nu$ on $\Omega$ such that $\mu(\Omega)=\nu(\Omega)$, $c$ a convex function.
 $T: \Omega \rightarrow \Omega$ such that $\begin{cases} & \nu(X)=\mu(T^{-1}(X)) \text{for any Borel (i.e. measurable) subset $X$ of $\Omega$}\\  & \int_{\Omega} c(x, T(x)) d \mu \text { is minimal }\end{cases}$\\ is called the optimal transport map from $\mu$ to $\nu$.
We are interested in $\mu$ continuos and $\nu$ discrete.
\end{frame}
\begin{frame}{Optimal Transport map II}
The Voronoi diagram Vor $(P)$ is the partition of $\mathbb{R}^{d}$ into the subsets $\operatorname{Vor}\left(p_{i}\right)$ defined by :
$\operatorname{Vor}\left(p_{i}\right):=\left\{x\|\| x-p_{i}\left\|^{2}<\right\| x-p_{j} \|^{2} \quad \forall j \neq i\right\}$
the power diagram Pow$_{W}(P)$ is the partition of $\mathbb{R}^{d}$ into the subsets Pow$_{W}\left(p_{i}\right)$ defined by:\\
$\operatorname{Pow}_{W}\left(p_{i}\right):=\left\{x \mid\left\|x-p_{i}\right\|^{2}-w_{i}<\left\|x-p_{j}\right\|^{2}-w_{j} \quad \forall j \neq i\right\}$\\
the map $T_{W}$ defined by $\forall i, \forall p \in \operatorname{Pow}_{W}\left(p_{i}\right), T_{W}(p)=p_{i}$ is called the assignment defined by the power diagram $\operatorname{Pow}_{W}(P)$ \textbf{and is an optimal transport map}.
\end{frame}
\begin{frame}
Knowing this we can approximate the optimal transport map in the following algorithm:\\
Data: Two tetrahedral meshes $M$ and $M^{\prime}$, and $k$ the desired number of vertices in the result\\
Result: A tetrahedral mesh $G$ with $k$ vertices and a pair of points $p_{i}^{0}$ and $p_{i}^{1}$ attached to each vertex. Transport is parameterized by time $t \in[0,1]$ with $p_{i}(t)=(1-t) p_{i}^{0}+t p_{i}^{1}$.\\
(1) Sample $M^{\prime}$ with a set $Y$ of $k$ points\\
(2) Compute the weight vector $W$ that realizes the optimal transport between $M$ and $Y$\\
(3) Compute $E=\operatorname{Del}(Y) \mid M^{\prime}$ and $F=\operatorname{Pow}_{W}(Y) \mid M $ $\operatorname{Tets}(\mathrm{G}) \leftarrow E \cap F$ where$ \operatorname{Del}$ is the Delaunay triangulation.\\
(4) Foreach $i \in[1 \ldots k],\left(p_{i}\right)^{0} \leftarrow \operatorname{centroid}\left(\operatorname{Pow}_{W}\left(y_{i}\right) \cap M\right) $ , $\quad\left(p_{i}\right)^{1} \leftarrow y_{i}$\\
\end{frame}
\begin{frame}{Problems}
There are two main problems with this approach:
\begin{itemize}
\item Step 3 may modify substantially the two meshes, so they may end up having different volumes (this howewer can be resolved by using more tetrahedras and rescaling at the end)
\item Let A and B the two meshes obtained with $p_{i}^{0}$ and $p_{i}^{1}$, then the map $f(t)=t*A+(1-t)*B$ will preserve volume with only very restrictive assumptions.
\end{itemize}
\end{frame}
\begin{frame}
\begin{theorem}
The volume is preserved for every $t\in [0,1]$ iff $\sum_{i=1}^{M} det(B_{i})(Tr(B_{i}^{-1}A_{i})-3)=0$  and $\sum_{i=1}^{M} det(A_{i})(Tr(A_{i}^{-1}B_{i})-3)=0$
\end{theorem}
A weaker version also holds
\begin{theorem}
$|Vol(t*A+(1-t)*B)-Vol(B)|\le \frac{4}{27}|\sum_{i=1}^{M} det(B_{i})(Tr(B_{i}^{-1}A_{i})-det(A_{i})Tr(A_{i}^{-1}B_{i}))|+\frac{1}{2}|\sum_{i=1}^{M} det(A_{i})(Tr(A_{i}^{-1}B_{i})-3)|
\end{theorem}

\end{frame}
\end{document}