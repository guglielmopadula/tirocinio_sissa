%%%%%%%%%%%%%%%%%%%%%%%%%%%%%%%%%%%%%%%%%
% baposter Portrait Poster
% LaTeX Template
% Version 1.0 (15/5/13)
%
% Created by:
% Brian Amberg (baposter@brian-amberg.de)
%
% This template has been downloaded from:
% http://www.LaTeXTemplates.com
%
% License:
% CC BY-NC-SA 3.0 (http://creativecommons.org/licenses/by-nc-sa/3.0/)
%
%%%%%%%%%%%%%%%%%%%%%%%%%%%%%%%%%%%%%%%%%

%----------------------------------------------------------------------------------------
%	PACKAGES AND OTHER DOCUMENT CONFIGURATIONS
%----------------------------------------------------------------------------------------

\documentclass[b0paper,portrait]{baposter}

\usepackage[font=small,labelfont=bf]{caption} % Required for specifying captions to tables and figures
\usepackage{booktabs} % Horizontal rules in tables
\usepackage{relsize} % Used for making text smaller in some places
\usepackage{amsfonts}
\usepackage{amsmath, bm}
\usepackage{hyperref}
\usepackage{pgfplots}
\usepackage{graphicx}
%\usepackage{wrapfig}
\usepackage{multicol}
\usepackage{array}
\usepackage{subfig}
\usepackage{url}
\usepackage[export]{adjustbox}

%\usepgfplotslibrary{patchplots}
%\pgfplotsset{compat=newest}

\graphicspath{{figures/}} % Directory in which figures are stored

\definecolor{bordercol}{RGB}{40,40,40} % Border color of content boxes
\definecolor{headercol1}{RGB}{171,205, 239}%{186,215,230} % Background color for the header in the content boxes (left side)
\definecolor{headercol2}{RGB}{192, 192, 192}%{80,80,80} % Background color for the header in the content boxes (right side)
\definecolor{headerfontcol}{RGB}{0,0,0} % Text color for the header text in the content boxes
\definecolor{boxcolor}{RGB}{186,215,230} % Background color for the content in the content boxe
%\definecolor{airforceblue}{rgb}{0.36, 0.54, 0.66}


\definecolor{bluemathlab}{HTML}{065895}
\definecolor{orangemathlab}{HTML}{F79A25}
\newcommand{\highlight}[1]{\textbf{\color{bluemathlab}#1}}
\newcommand{\highlightB}[1]{\textbf{\color{black!15!orangemathlab}#1}}
\newcommand{\rbnics}{\highlightB{\texttt{RB}}\highlight{\texttt{niCS}}}
\newcommand{\fenics}{\texttt{FEniCS}}
\newcommand{\ped}[1]{_{\mathrm{#1}}}
\newcommand{\up}[1]{^{\mathrm{#1}}}
\renewcommand{\r}{\mathbf{r}}
\renewcommand{\d}{\mathrm{d}}
\newcommand{\x}{\times}
\newcommand{\dund}[1]{\underline{\underline{#1}}}
\newcommand{\und}[1]{\underline{#1}}
\newcommand{\mmu}{\boldsymbol\mu}

\def\Put(#1,#2)#3{\leavevmode\makebox(0,0){\put(#1,#2){#3}}}
\usepackage[procnames]{listings}

\definecolor{keywords}{RGB}{255,0,90}
\definecolor{comments}{RGB}{0,0,113}
\definecolor{red}{RGB}{160,0,0}
\definecolor{green}{RGB}{0,150,0}
\definecolor{mygray}{rgb}{0.5,0.5,0.5}

\lstset{language=Python, 
        basicstyle=\ttfamily\small, 
        keywordstyle=\color{keywords},
        commentstyle=\color{comments},
        stringstyle=\color{red},
        showstringspaces=false,
        identifierstyle=\color{green},
        procnamekeys={def,class},
		breaklines=true,
%    	postbreak={\ensuremath{\color{red}\hookrightarrow}},
  numbers=left,                    % where to put the line-numbers; possible values are (none, left, right)
  numbersep=-5pt,                   % how far the line-numbers are from the code
  numberstyle=\tiny\color{mygray}, % the style that is used for the line-numbers
}

\begin{document}

\background{ % Set the background to an image (background.pdf)
\begin{tikzpicture}[remember picture,overlay]
\draw (current page.north west)+(-2em,2em) node[anchor=north west]
{\includegraphics[height=1.1\textheight]{background.png}};
\end{tikzpicture}
}

\begin{poster}{
grid=false,
borderColor=bordercol, % Border color of content boxes
headerColorOne=headercol1, % Background color for the header in the content boxes (left side)
headerColorTwo=headercol2, % Background color for the header in the content boxes (right side)
headerFontColor=headerfontcol, % Text color for the header text in the content boxes
boxColorOne= white,%boxcolor, % Background color for the content in the content boxes
%headershape=rounded,%right, % Specify the rounded corner in the content box headers
headerfont=\Large\sf\bf, % Font modifiers for the text in the content box headers
textborder=rectangle,
background=user,
headerborder=closed, % Change to closed for a line under the content box headers
boxshade=plain,
columns=6,
headerheight=0.1\textheight
}
{%
%\hspace{.2cm}
}%{\includegraphics[height=3.5em]{erc}\hspace{.2cm}\includegraphics[height=7.5em]{aroma-logo}}
%
%----------------------------------------------------------------------------------------
%	TITLE AND AUTHOR NAME
%----------------------------------------------------------------------------------------
%
{\\\vspace{0.25cm} {\huge\bf Machine Learning and Optimal Transport \\ \vspace{0.25cm} for shape parametrisation}} % Poster title
{\vspace{0.25cm} Guglielmo Padula, Francesco Romor, Giovanni Stabile, Nicola Demo, Gianluigi Rozza \\ % Author names
{\smaller \vspace{0.25cm} Mathematics Area, mathLab, SISSA, International School of Advanced Studies, Trieste, Italy}
} % Author email addresses
{
} % University/lab logo
\Put(70,2200){\includegraphics[height=6em]{logo_sissa_cerchio.png}}
\Put(0,2200){\includegraphics[height=6em]{logo-mathlab_no_borders}}
\Put(665,2200){\includegraphics[height=6em]{logo_uni}}
\Put(735,2200){\includegraphics[height=6em]{logo_dssc}}

%----------------------------------------------------------------------------------------

\headerbox{Introduction}{name=introduction,column=0,row=0,span=6}{
\begin{itemize} 
\item generative models of shape optimization for complex geometries with an elevated number of parameters; the objective is to learn the shape using a minor number of parameters, for example for modeling naval hull, and creating new geometries similar to real data, as creating new geometries can be costly.
\item Semidiscrete Optimal transport is used to obtain an optimal transport map to find some intermediate geometries with some regularity constraints. We show results regarding Variational Autoencoders, Semi-Discrete Optimal Transport applied to simple geometries. The extension of Semi Discrete Optimal Transport to three meshes will also be discussed.
\end{itemize}
}


%----------------------------------------------------------------------------------------

\begin{posterbox}[name=volume,below=introduction,span=6,column=0]{1
    - Volume of a Tetrahedral Mesh }
%\vspace{-0.5cm}
Consider a tetrahedal mesh $M$ composed of $N$ tetrahedras.
Let $M_{i}$ the i-th tetrahedra of $M$ and $a^{(i)},b^{(i)},c^{(i)},d^{(i)}$ it's edges represented as column vectors of $\mathbb{R}^{3}$. It can be proven that $Vol(M_{i})=\frac{|det(A_{i})|}{6}$ where $A_{i}$ is the matrix $[a^{(i)}-d^{(i)}\quad b^{(i)}-d^{(i)}\quad c^{(i)}-d^{(i)} $].\\

%$$A_{i}=\begin{array}{|ccc|}
%a^{(i)}_{1}-d^{(i)}_{1} & b^{(i)}_{1}-d^{(i)}_{1} & c^{(i)}_{1}-d^{(i)}_{1} \\
%a^{(i)}_{2}-d^{(i)}_{2}& b^{(i)}_{2}-d^{(i)}_{2} & c^{(i)}_{2}-d^{(i)}_{2} \\
%a^{(i)}_{3}-d^{(i)}_{3} & b^{(i)}_{3}-d^{(i)}_{3} & c^{(i)}_{3}-d^{(i)}_{3} \end{array}$$
It follows that $$Vol(M)=\sum\limits_{i=1}^{N} Vol(M_{i})=\sum\limits_{i=1}^{N}\frac{|det(A_{i})|}{6}$$
\end{posterbox}

%----------------------------------------------------------------------------------------

\begin{posterbox}[name=otm,below=volume,span=6,column=0]{2
    - Optimal Transport }
%\vspace{-0.5cm}
Given $\Omega$ a Borel set and two measures $\mu$ and $\nu$ on $\Omega$ such that $\mu(\Omega)=\nu(\Omega)$, $c$ a convex function $T: \Omega \rightarrow \Omega$ such that $$\begin{cases} \nu(X)=\mu(T^{-1}(X)) \quad \text{for any Borel (i.e. measurable) subset $X$ of $\Omega$}\\  \int_{\Omega} c(x, T(x)) d \mu \quad \text { is minimal }\end{cases}$$\\ is called the \textbf{optimal transport} map from $\mu$ to $\nu$.\\
We are interested in $\mu$ continuos and $\nu$ discrete. It this case we talk of Semi Discrete Optimal Transport.\\
The power diagram $\operatorname{Pow}_{W}\left(p_{i}\right)_{W}(P)$ is the partition of $\mathbb{R}^{d}$ into the subsets $\operatorname{Pow}_{W}\left(p_{i}\right):=\left\{x \mid\left\|x-p_{i}\right\|^{2}-w_{i}<\left\|x-p_{j}\right\|^{2}-w_{j} \quad \forall j \neq i\right\}.$\\
The map $T_{W}$ defined by $\forall i, \forall p \in \operatorname{Pow}_{W}\left(p_{i}\right), T_{W}(p)=p_{i}$ is called the assignment defined by the power diagram $\operatorname{Pow}_{W}(P)$ and is an optimal transport map.
\\Knowing this, we can approximate the optimal transport map with the following algorithm:\\
$\null$\\
Data: Two tetrahedral meshes $M$ and $M^{\prime}$, and $k$ the desired number of vertices in the result\\
Result: A tetrahedral mesh $G$ with $k$ vertices and a pair of points $p_{i}^{0}$ and $p_{i}^{1}$ attached to each vertex. Transport is parameterized by time $t \in[0,1]$ with $p_{i}(t)=(1-t) p_{i}^{0}+t p_{i}^{1}$.\\
(1) Sample $M^{\prime}$ with a set $Y$ of $k$ points\\
(2) Compute the weight vector $W$ that realizes the optimal transport between $M$ and $Y$\\
(3) Construct $E=Del(Y)$ where $Del$ it the Delaunay Triangulation.\\
(4) For each $i \in[1 \ldots k],\left(p_{i}\right)^{0} \leftarrow \operatorname{centroid}\left(\operatorname{Pow}_{W}\left(y_{i}\right) \cap M\right) $ , $\quad\left(p_{i}\right)^{1} \leftarrow y_{i}$\\
G will be the mesh defined by the topology of $E$ with the pair of points $(p_{i})^{0}$,$(p_{i})^{1}$.
It can be proven that the linear map does not clash particles. This algorithm is implemented in the library Geogram created by Bruno Levy.
\end{posterbox}
\begin{posterbox}[name=otm2,below=otm,span=6,column=0]{3a
    - Extension of Optimal Transport: volume preserviness in intermediate times }
%\vspace{-0.5cm}
Let $M^{(0)}$ and $M^{(1)}$ the results of the Optimal Transport. In theory, $M^{(0)}\neq M$ because of the intersection with the power diagram so $M^{(0)}\neq M$. For similar reasoning, $M^{(1)}\neq M^{\prime}$. However, with $k$ large, it can be shown experimentally that the meshes are similar. Let us suppose we are in this case and so $Vol(M^{(0)})=Vol(M^{(1)})$. As the volume of a mesh is a scaled sum of determinant and determinant are non linear functions, in general, we have $Vol(tM^{(0)}+(1-t)M^{(1)})\neq Vol(M^{(0)})$ so a simple linear map cannot preserve volumes in intermediate times.
Let $M_{i}^{(0)}$ and $M_{i}^{(1)}$, $A_{i}^{(0)}$ and $A_{i}^{(1)}$ like in section 1. From the property that the linear map does not clash particles, we have that $det(A_{i}^{(0)})det(A_{i}^{(1)})>0 \forall i \in 1...N $. Otherwise, as determinant is a continuous function, we would have that $\exists t^{*}$ s.t $0=det(t^{*}A_{i}^{(0)}+(1-t^{*})A_{i}^{(1)})=6*Vol(t^{*}M_{i}^{(0)}+(1-t^{*})M_{i}^{(1)})$ so a tetrahedra disappear and this cannot happen because of the non clashing particles property. So we can cut the absolute value out and obtain $Vol(tM^{(0)}+(1-t)M^{(1)})=\sum \limits_{i=1}^{N} Vol(tA_{i}^{(0)}+(1-t)A_{i}^{(1)})=\sum \limits_{i=1}^{N} \frac{Det(tA_{i}^{(0)}+(1-t)A_{i}^{(1)})}{6}$. The volume in this case is just a third grade polynomial of grade $3$, and can also be calculated analitycally knowing $M^{(0)}$ and $M^{(1)}$.
With this motivations, we define another map for optimal transport $$\phi_{M^{(0)}M^{(1)}} (t)= {Vol(M^{(0)})}^{\frac{1}{3}}\frac{tM^{(1)}+(1-t)M^{(0)}}{Vol(tM^{(1)}+(1-t)M^{(0)})^{\frac{1}{3}}}$$
This map is an optimal transport map as $\phi_{M^{(0)}M^{(1)}}(0)=M^{(0)}$ and $\phi_{M^{(0)}M^{(1)}}(1)=M^{(1)}$ and preserves volume in intermediate times by definition.
\end{posterbox}
\begin{posterbox}[name=otm3,below=otm2,span=6,column=0]{3b
    - Extension of Optimal Transport: extension to three meshes }
Let $M$, $M^{\prime}$, $M^{\prime\prime}$ three meshes. Let us suppose we calculate the OTM from $M$ to $M^{\prime}$ and from $M^{\prime\prime}$ selecting  $M^{\prime}$ to be sampled in the same way in both transports, so the resulting mesh will be the same (because in the algorithm does not depend on the other). So let $M^{(0)}$, $M^{(1)}$ and $M^{(2)}$ the approximation of $M$, $M^{\prime}$, $M^{\prime\prime}$ respectively. 
So with the same assumption as before we can define a new transport map $$\phi_{M^{(0)}M^{(1)}M^{(2)}}(t)=\begin{cases}
Vol(M^{(1)})^{\frac{1}{3}}\frac{2tM^{(1)}+(1-2t)M^{(0)}}{Vol(2tM^{(1)}+(1-2t)M^{(0)})^{\frac{1}{3}}} & 0 \le t \le \frac{1}{2} \\
Vol(M^{(1)})^{\frac{1}{3}}\frac{(2t-1)M^{(2)}+(2-2t)M^{(1)}}{Vol((2t-1)M^{(2)}+(2-2t)M^{(1)})^{\frac{1}{3}}} & \frac{1}{2} \le t \le 1
\end{cases}$$
that goes from $M^{(0)}$ to $M^{(1)}$ and then from $M^{(1)}$ to $M^{(2)}$.  
\end{posterbox}

\begin{posterbox}[name=vae,below=otm3,span=6,column=0]{4
    - Variational Autoencoder }
 A Variational Autoencoder is a neural network composed of two main components, an encoder $E:X\rightarrow Z$ and a decoder $D:Z\rightarrow X$ where usually $dim(Z)<<dim(X)$. If it is possible to sample from $Z$ after $E$ and $D$ are trained it is possible to sample from $X$ by sampling from $Z$ and applying $D$. $Z$ is also useful for dimensionality reduction. 
\end{posterbox}
\begin{posterbox}[name=results,below=vae,span=6,column=0]{5
    - Preliminary results and future work}
Summary of our results:
\begin{itemize}
\item We are able to do volume preserving deformation between three meshes using Optimal Trasport Map with meshes of convex polyhedras, our next step is to generalize it for more than three meshes, which is nontrivial (as $M_{0}$ and $M_{2}$ are nonsampled), and also to test less regular meshes.
\item We are able to sample parallelepipeds centered in $0$ with volume $1$. However, the space $Z$ is too much sparse even for these simple meshes, so we are planning to move to Generative Adversarial Networks. 
\end{itemize}
\end{posterbox}
\begin{posterbox}[name=bibliography,below=results,span=6,column=0]{Bibliography and Software References}
\begin{itemize}
\item Geogram, \url{http://alice.loria.fr/software/geogram/doc/html/index.html}
\item A numerical algorithm for $L_{2}$ semi-discrete optimal transport in 3D, Bruno Levy, arXiv, 2014
\item Variational Autoencoders for Deforming 3D Mesh Models, Qingyang Tan, arXiv, 2018
\end{itemize}
\end{posterbox}

\end{poster}

\end{document}
