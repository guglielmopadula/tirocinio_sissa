\documentclass{article}

% Language setting
% Replace `english' with e.g. `spanish' to change the document language
\usepackage[english]{babel}

% Set page size and margins
% Replace `letterpaper' with `a4paper' for UK/EU standard size
\usepackage[letterpaper,top=2cm,bottom=2cm,left=3cm,right=3cm,marginparwidth=1.75cm]{geometry}

% Useful packages
\usepackage{amsmath}
\usepackage{graphicx}
\usepackage[colorlinks=true, allcolors=blue]{hyperref}

\title{Title}
\author{Guglielmo}

\begin{document}
\maketitle

\section{Preface}
 Our line of work consists in studying \textbf{generative models} for shape optimization of complex geometries with a large number of parameters; the objective is also to reduce the number of relevant geometrical parameters, for example for modeling naval hulls, and creating new artificial geometries similar to real data, as there are non-generative techniques for creating new real geometries (for example FFD) but using them can be costly. Our data is artificially created data using FFD using a real bulbus model provided by Fincantieri. We study mainly two class of models: VAE's and GAN's, using both classical neural networks and geometric neural networks. With at a least one model for every type we arrive at a relative reconstruction error of $0.002$.


\end{document}
