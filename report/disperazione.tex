 \documentclass[11pt]{article}

\usepackage{sectsty}
\usepackage{graphicx}
\usepackage{amsfonts}
\usepackage{amsmath}

% Marg\documentclass[11pt]{article}

\usepackage{sectsty}
\usepackage{graphicx}
\usepackage{amsfonts}
\usepackage{amsmath}
\usepackage{amsthm}
\newtheorem{theorem}{Theorem}


\title{Disperazione}
\author{Guglielmo Padula}
\DeclareMathOperator*{\argmax}{arg\,max}
\DeclareMathOperator*{\argmin}{arg\,min}
\begin{document}
\maketitle
\begin{theorem}
Let $\phi:\mathbb{R}^{3}\rightarrow \mathbb{R}^{3}$ differentiable such that $Vol(\phi(B))=Vol(B) \forall B\subseteq \mathbb{R}^{3}$. Then 
$|Det(J_{\phi})|=1$. 
\end{theorem}
\begin{proof}
$Vol(\phi(B))=\int_{\phi(B)} dx=\int_{B}|Det(J_{\phi})|dx$\\
Also $Vol(B)=\int_{B}dx$\\
So we get $\int_{B}dx=\int_{B}|Det(J_{\phi})|dx$ $\forall B$,which means $|Det(J_{\phi})|=1\forall x$.
\end{proof}

\begin{theorem}
Let $\phi:\mathbb{R}^{3}\rightarrow \mathbb{R}^{3}$ and $C^{1}$ such that $Vol(\phi(B))=Vol(B) \forall B\subseteq \mathbb{R}^{3}$. Then $\phi(x)=Ax+b$ with $det(A)=1$ or $det(A)=-1$.
\end{theorem}
\begin{proof}
By theorem 1 we get $Det(J_{\phi})|=1$. Howewer now $J_{phi}$ is continous and so we get $det(J_{phi})=1 \forall x $ or $det(J_{phi})=-1 \forall x$, then basic integration brings $\phi(x)=
\end{document}